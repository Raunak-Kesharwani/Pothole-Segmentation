\documentclass[12pt, a4paper]{report}
\usepackage{graphicx}
\usepackage{hyperref}
\usepackage{listings}
\usepackage{color}
\usepackage{geometry}
\usepackage{booktabs}
\usepackage{float}
\usepackage{titlesec}

\geometry{
 a4paper,
 total={170mm,257mm},
 left=20mm,
 top=20mm,
}

\definecolor{codegreen}{rgb}{0,0.6,0}
\definecolor{codegray}{rgb}{0.5,0.5,0.5}
\definecolor{codepurple}{rgb}{0.58,0,0.82}
\definecolor{backcolour}{rgb}{0.95,0.95,0.92}

\lstdefinestyle{mystyle}{
    backgroundcolor=\color{backcolour},   
    commentstyle=\color{codegreen},
    keywordstyle=\color{magenta},
    numberstyle=\tiny\color{codegray},
    stringstyle=\color{codepurple},
    basicstyle=\ttfamily\footnotesize,
    breakatwhitespace=false,         
    breaklines=true,                 
    captionpos=b,                    
    keepspaces=true,                 
    numbers=left,                    
    numbersep=5pt,                  
    showspaces=false,                
    showstringspaces=false,
    showtabs=false,                  
    tabsize=2
}

\lstset{style=mystyle}

\title{\textbf{AI-Powered Pothole Segmentation and Reporting System}\\
\large A Comprehensive Technical Report}
\author{Developed by [Your Name/Team Name]}
\date{\today}

\begin{document}

\maketitle

\begin{abstract}
Road infrastructure maintenance is a critical challenge in urban planning, with potholes being a primary cause of accidents and vehicle damage. Traditional methods of manual inspection are slow, costly, and prone to human error. This project implements an automated "Pothole Segmentation & Reporting System" utilizing deep learning (YOLOv8) for real-time detection and semantic segmentation of road damage. The system features a Progressive Web App (PWA) client for citizens to report issues, a role-based administrative dashboard for managing repairs, and a robust cloud backend (Supabase) for data persistence. This report details the system architecture, machine learning methodology, implementation challenges, and evaluation results, demonstrating a solution that significantly reduces the time from detection to repair deployment.
\end{abstract}

\tableofcontents
\newpage

\chapter{Introduction}
\section{Background}
Poor road conditions are a global issue, but they remain particularly acute in developing nations. Potholes not only disrupt traffic flow but pose severe safety risks. The gap between a citizen spotting a pothole and the authorities fixing it is often widened by inefficient reporting mechanisms.

\section{Problem Statement}
Current reporting systems often rely on manual phone calls or generic web forms lacking precise location data or severity assessment. Authorities lack a centralized, prioritized view of road defects, leading to resource mismanagement.

\section{Proposed Solution}
We propose an end-to-end digital ecosystem:
\begin{itemize}
    \item \textbf{Citizen App}: A mobile-first web app that uses the camera to detect potholes in real-time, calculates severity, and auto-tags GPS coordinates.
    \item \textbf{AI Engine}: A computer vision model capable of distinguishing potholes from shadows or patches with high precision.
    \item \textbf{Admin & Worker Portals}: A dedicated workflow for assigning repairs to field workers and tracking completion with visual proof.
\end{itemize}

\chapter{System Architecture}

\section{High-Level Overview}
The system follows a modern microservices-inspired architecture, separating the frontend client, the inference API, and the database layer.

\begin{figure}[H]
    \centering
    % \includegraphics[width=0.8\textwidth]{architecture_diagram.png}
    \caption{System Architecture Diagram (Conceptual)}
\end{figure}

\section{Technology Stack}
\begin{itemize}
    \item \textbf{Frontend}: React.js (Vite), TypeScript, Tailwind CSS, Framer Motion.
    \item \textbf{Backend API}: Python FastAPI, ONNX Runtime (for inference).
    \item \textbf{Database & Auth}: Supabase (PostgreSQL, Row Level Security).
    \item \textbf{Machine Learning}: YOLOv8-Seg (trained on Pothole-600 dataset).
\end{itemize}

\section{Data Flow}
1. \textbf{Capture}: User takes a photo via the React App.
2. \textbf{Inference}: Image is sent to the FastAPI backend.
3. \textbf{Analysis}: Model returns a segmentation mask and confidence score.
4. \textbf{Processing}: Frontend calculates the defect area (pixels) and severity.
5. \textbf{Storage}: Image uploaded to Supabase Storage; Metrics saved to SQL Database.
6. \textbf{Action}: Admin assigns the report to a worker via the Dashboard.

\chapter{Machine Learning Methodology}

\section{Model Selection: YOLOv8-Seg}
We selected YOLOv8 (You Only Look Once) Segmentation variant due to its balance of speed and accuracy. Unlike bounding-box detection, segmentation provides the exact shape of the pothole, allowing for pixel-area calculation which is crucial for estimating material costs for repair.

\section{Training Process}
\subsection{Dataset}
The model was trained on a curated dataset of 2,000+ annotated road images containing varying lighting conditions, wet surfaces, and different pothole types.

\subsection{Hyperparameters}
\begin{itemize}
    \item \textbf{Epochs}: 100
    \item \textbf{Batch Size}: 16
    \item \textbf{Optimizer}: AdamW
    \item \textbf{Learning Rate}: 0.001 (with Cosine Decay)
\end{itemize}

\section{Performance Metrics}
The final model achieved:
\begin{itemize}
    \item \textbf{mAP@50}: 0.894 (Mean Average Precision)
    \item \textbf{Inference Time}: $\approx$ 150ms on CPU (ONNX Runtime)
\end{itemize}

\chapter{Implementation Details}

\section{Frontend Development (React)}
The frontend is designed as a PWA. Key challenges included handling camera streams across different mobile browsers and optimizing the base64 image handling to prevent browser memory crashes (QuotaExceededError).

\begin{lstlisting}[language=JavaScript, caption=Prediction Context Logic]
// Optimization to prevent LocalStorage crashes
const toPersist = predictions.map(p => ({
  ...p,
  imageDataUrl: null, // Don't persist heavy images
  metrics: p.metrics
}));
localStorage.setItem('history', JSON.stringify(toPersist));
\end{lstlisting}

\section{Backend API (FastAPI)}
The backend exposes a single endpoint `/predict`. It decodes the uploaded image, pre-processes it (resize/normalization), runs the ONNX session, and post-processes the output masks.

\section{Database Schema (Supabase)}
We utilize PostgreSQL's relational capabilities.
\begin{itemize}
    \item \textbf{users}: Extends standard auth with roles (admin/citizen/worker).
    \item \textbf{reports}: Stores the pothole data, linked to users.
    \item \textbf{tasks}: Joins reports to workers for assignment tracking.
\end{itemize}

\chapter{Role-Based Workflows}

\section{Citizen Workflow}
Citizens can simply "Point and Shoot". The app provides immediate feedback: "High Severity Pothole Detected". The report is auto-submitted if they are logged in.

\section{Administrator Workflow}
Admins access a secure dashboard. They can filter reports by severity ("Critical" first). A dropdown allows assigning verified workers to specific locations.

\section{Worker Workflow}
Field workers have a simplified view: "My Tasks". They navigate to the GPS location, fix the pothole, and must upload a "After" photo to close the ticket.

\chapter{Results and Discussion}

\section{Detection Accuracy}
The system successfully identifies potholes in 90\% of test cases. False positives occasionally occur with dark manhole covers or heavy shadows, which are mitigated by the confidence threshold filter ($>0.25$).

\section{System Performance}
End-to-end latency (Capture to Result) is under 2 seconds on 4G networks. The utilization of ONNX Runtime allows the model to run efficiently even on lower-cost server instances without simplified GPUs.

\chapter{Conclusion and Future Work}

\section{Conclusion}
This project successfully demonstrates the viability of AI-driven civic engagement tools. By automating the detection and quantification of road defects, we reduce the administrative burden on city planners and empower citizens to be active participants in infrastructure maintenance.

\section{Future Improvements}
\begin{itemize}
    \item \textbf{Mobile Edge Inference}: Running the model directly in the browser (TF.js) to remove server dependency.
    \item \textbf{Volumetric Estimation}: Using depth estimation models to calculate pothole depth.
    \item \textbf{Offline Mode}: Queuing reports when no internet is available and syncing later.
\end{itemize}

\appendix
\chapter{Appendix A: API Reference}
\textbf{POST /predict}
\begin{itemize}
    \item Input: `multipart/form-data` (image file)
    \item Output: JSON `{ is_pothole: boolean, confidence: float, mask: base64 }`
\end{itemize}

\end{document}
